\documentclass[a4paper, oneside, 11pt]{scrartcl}

% Get the necessary Packages
% Set to english language and utf8.
\usepackage[english]{babel}
\usepackage[utf8]{inputenc}

% Some packages for symbols we need within the tutorial.
\usepackage{dingbat}
\usepackage{marvosym}

% For the sourcecode.
\usepackage{listings}

% Enable Jan's highlighting
% Usage:
% %% preamble
% \usepackage{listings} %% correct name?
\usepackage{lstide}
% \lstset{tabsize=2,captionpos=b,style=default,}
%
%
% %% main
%     \begin{lstlisting}[style=eclipse-java,gobble=6,caption={Simple micro-benchmark in Java}]
%       for (int i = 0; i < 1000; i++) {
%         tin = currentTime();
%         benchmarkedOperation // testtext
%         tout = currentTime();
%       }
%     \end{lstlisting}

% For the links etc.
\usepackage{hyperref} % avh: removed [pdfborder={0 0 0}]

% For the pdf-graphics.
\usepackage{graphicx}

% The steamroller tactics to fix figures and so on.
\usepackage{float}

% This is for tables which are to long to be shown on one page.
\usepackage{longtable}

% This package is for the directory tree structures
\usepackage{dirtree}
\renewcommand*\DTstylecomment{\footnotesize\normalfont\itshape\rmfamily}
\renewcommand*\DTstyle{\footnotesize\sffamily}

% We need this package for some color within the document.
\usepackage{color}

\usepackage[sort,compress,numbers]{natbib} %round,authoryear

% compactitem, compactenum, ...
\usepackage{paralist}


% This is the package for the margin-nodes.
\usepackage[color=white, bordercolor=white]{todonotes}

\usepackage{amsfonts}
\usepackage{setspace}
\usepackage{ae,aecompl}

\usepackage[automark]{scrpage2}

\usepackage[margin=0.5cm,indention=0em,font={small},labelfont={sf,small},format=hang]{caption}

\usepackage[hang,sf]{subfigure}
\subfigcapmargin=1em

%\usepackage{scrhack}

% Only some new commands.
\newcommand{\Kieker}{\textit{Kieker}}
\newcommand{\home}{$\sim$}

% Set title and everything.
\title{Tutorial for \Kieker:\\ Monitoring and Analysis of Software Behavior}

% Here we go.
\begin{document}
\maketitle
\tableofcontents

\newpage

\section{Overview}
\subsection{What is \Kieker?}
\Kieker\ is a framework\footnote{A framework is sort of a library which provides specific and extended functionality} which allows programmers and software engineers the monitoring and analysis of program flows and the runtime behavior of java applications. Normal (``plain'') java applications can be arranged with the framework as well as server based java web applications. The framework itself aims to provide an easy managable and maintanable piece of software, which can be included uncomplicated into existing software projects. While \Kieker\ analyzes the own sourcecode reliably, it causes itself only very less overhead during monitoring.\\
\Kieker\ can be used to put whole method calls on a watch, but single statements (e.g. a = a + 1) as well.
\begin{figure}[H]
	\begin{center}
		\includegraphics[width=1.0\textwidth]{kiekerComponentDiagram.pdf}
		\label{image:componentdiagram}
		\caption{\Kieker\ component diagram}
	\end{center}
\end{figure}
As can be seen in figure \ref{image:componentdiagram}, the framework consists mainly of two big parts:
\begin{itemize}
\item \textbf{\KiekerMonitoring}\\
This is the part which is responsible for the logging and the recording of the program behavior. The result of this component are the recorded informations which can then be written into different output streams, like for example into files or into a database.
\item \textbf{\KiekerAnalysis}\\
This part is responsible for the evaluation and visualization of the recorded information. It uses the files (or general any collected data which is available as monitoring records) for the analysis and to produce graphs (e.g. Component-Dependency-Graph).
\end{itemize}

\subsection{What ist the purpose of this tutorial?}
In this tutorial, we will take a closer look at both, the \textbf{\KiekerMonitoring}- and the \textbf{\KiekerAnalysis}-part. That means, we will describe on the one hand how \KiekerMonitoring\ can be used to mark parts of the own sourcecode for \Kieker\ and to let them execute under surveilance, so that the recorded information can be saved in files. On the other hand we will use \KiekerAnalysis\ to visualize our recorded data.\\
We will show how to create and execute a simple example before we go deeper into the parts of the framework.

\section{Quickstart}
\subsection{Downloading and installing \Kieker}
For the monitoring and analysis of the source code, it is necessary to download the \Kieker\ binaries from \KiekerDownloadUrl\ first. Once that is done, the content of the zip- respectively the tar.gz-file should be extracted to any directory, for example ``$\sim$/kieker'' (under Linux) or ``c:$\backslash$program files$\backslash$kieker'' (under Windows).\\
That is already enough for the ``installation'' of \Kieker.

\subsection{Monitoring}
Now for the creation of a simple example for the use of the \Kieker\ framework (note: The example is available on the website of \Kieker). It is recommended to create a new working directory (e.g. $\sim$/example) with the following subdirectories:
\begin{itemize}
  \item src (For the sourcecode files)
  \item lib (For the libraries and needed jar-files)
  \item META-INF (For the configuration files of \Kieker)
  \item build (For the builded class files of Java)
\end{itemize}
Before we start with the sourcedoe, we need to copy some files from the \Kieker\ directory to our own working directory\footnote{It would be possible to access the required files within the \Kieker\ directory, but the copying will make the compiling much more comfortable.}.
\begin{itemize}
  \item $\sim$/kieker/dist/kieker-tpmon-\version\_ctrl.jar to $\sim$/example/lib/kieker-tpmon-\version\_ctrl.jar
  \item $\sim$/kieker/dist/kieker-common-\version.jar to $\sim$/example/lib/kieker-common-\version.jar
  \item $\sim$/kieker/lib/commons-logging-\version.jar to $\sim$/example/lib/commons-logging-\version.jar
  \item $\sim$/kieker/META-INF/tpmon.properties.example to $\sim$/META-INF/\textbf{tpmon.properties}
  \item $\sim$/kieker/META-INF/log4j.properties.example to $\sim$/META-INF/\textbf{log4j.properties}
\end{itemize}
The last two files are configuration files, but for a quick start they are already configurated correctly.\\

We start with creating two directories for our packages:
\begin{itemize}
  \item $\sim$/example/src/mySimpleKiekerExample
  \item $\sim$/example/src/mySimpleKiekerExample/bookstoreTracing
\end{itemize}
In the last directory, we create three files: 
\begin{itemize}
  \item CRM.java
  \item Catalog.java
  \item Bookstore.java
\end{itemize}
The file ``Bookstore.java'' should contain of the lines showed in listing \ref{listing:Bookstore.java}.

\setJavaCodeListing

\lstset{caption=Bookstore.java, label=listing:Bookstore.java}
\lstinputlisting{source-example/src/Bookstore.java}
Listing \ref{listing:Catalog.java} shows the content of ``Catalog.java'' and listing \ref{listing:CRM.java} the content of ``CRM.java''.
\lstset{caption=Catalog.java, label=listing:Catalog.java}
\lstinputlisting{source-example/src/Catalog.java}
\lstset{caption=CRM.java, label=listing:CRM.java}
\lstinputlisting{source-example/src/CRM.java}

The monitoring itself is done manually. Although this is not the strength of \Kieker\ it is pretty good for a quick start.
\lstset{caption=Cutting from Bookstore.java, label=listing:cuttingBookstore}
\begin{lstlisting}
long tin = TpmonController.getInstance().getTime();
Bookstore.searchBook();
long tout = TpmonController.getInstance().getTime();

KiekerExecutionRecord e = KiekerExecutionRecord.getInstance("mySimpleKiekerExample.bookstoreTracing.Bookstore", "searchBook()", "sessionID", 0, tin, tout, "vnName", 0, 0);
TpmonController.getInstance().logMonitoringRecord(e);
\end{lstlisting}
In listing \ref{listing:cuttingBookstore} can be seen, how the monitoring itself is done. We use the \textit{TpmonController} to get the current time in nano seconds and remember the time before and after a specific method call (in this case: \textit{searchBook()})\footnote{The code between the timekeeping does not need to be a method call of course. It can be ``plain'' code or more than one method call as well.}. These informations are stored in the so called execution record. It gets:
\begin{itemize}
 \item The component (the class) in which the called method is.
 \item The called method.
 \item The session id. In this case we can use any string.
 \item The trace id of the current trace we want to record. Due to the fact, that we follow only one trace, this is zero in all recordings.
 \item The time before the sourcecode which should be measured.
 \item The time after the sourcecode which should be measured.
 \item The name of the current host. This is not very important in this case, because we have only one host. The name can be choosen freely.
 \item The eoi (execution order index). This tells \Kieker\ later the sequence of the different calls. It should be of course unique within a trace.
 \item The ess (execution stack size). This number tells \Kieker\ that the execution was started when the calling stack of the corresponding trace was just the ess.
\end{itemize}
For the moment we have to choose the eoi and ess manually. These numbers can be choosen later automaticaly by \Kieker\ of course.\\
Once this is done, we should be able to compile and execute the sourcecode:
\setBashListing
\begin{lstlisting}
nils@Laptop:~/example$ javac ./src/Bookstore.java ./src/Catalog.java ./src/CRM.java -classpath ./lib/kieker-tpmon-1.1_ctrl.jar -d ./build
nils@Laptop:~/example$ java -Dlog4j.configuration=META-INF/log4j.properties -classpath ./build/:./lib/kieker-tpmon-1.1_ctrl.jar:./lib/kieker-common-1.1.jar:./lib/commons-logging-1.1.1.jar mySimpleKiekerExample.bookstoreTracing.Bookstore
\end{lstlisting}
If everything worked correctly, there should now be a new directory named ``tpmon-20100605-115948636-UTC'' (just with other numbers) in the default temporary directory (under Linux this should be ``/tmp''). In this directory, there should be a file with the extension ``.dat'' with a content similar to the following:
\begin{lstlisting}
$1;1275745883934593403;-1;mySimpleKiekerExample.bookstoreTracing.Catalog.getBook(false);sessionID;0;1275745883931011663;1275745883933424540;vnName;1;1
$1;1275745883937096236;-1;mySimpleKiekerExample.bookstoreTracing.Catalog.getBook(false);sessionID;0;1275745883935003302;1275745883937075214;vnName;3;2
$1;1275745883937119354;-1;mySimpleKiekerExample.bookstoreTracing.CRM.getOffers();sessionID;0;1275745883934661568;1275745883937111043;vnName;2;1
$1;1275745883937128922;-1;mySimpleKiekerExample.bookstoreTracing.Bookstore.searchBook();sessionID;0;1275745883931007961;1275745883937123824;vnName;0;0 
\end{lstlisting}
These are the recorded informations from our sourcecode. This data can now be visualized with the help of \Kieker.

\subsection{Analysis}

\section{\KiekerMonitoring}
\subsection{Configuration}
\subsection{Probes}
\subsection{Writers}

\section{\KiekerAnalysis}
\subsection{Configuration}
\subsection{Readers}
\subsection{Consumers}

\section{Appendix}
\subsection{Example logs}
\subsection{Shortcut via ant}
\subsection{Libraries}
\begin{center}
\begin{longtable}{|p{0.4\textwidth}|p{0.5\textwidth}|}
\hline 
Filename & Description\\
\hline
\hline 
aspectjrt-1.6.11.jar & This jar-file contains the runtime library for AspectJ programs.\\
\hline 
aspectjtools-1.6.11.jar & This package contains the tools (the AspectJ Compiler and Browser) for AspectJ.\\
\hline 
aspectjweaver-1.6.11.jar & This jar contains the weaver-agent for the aspect-oriented-extension for Java named AspectJ.\\
\hline 
commons-cli-1.2.jar & Apache Commons CLI provides a simple API for working with the command line arguments and options.\\
\hline 
commons-logging-1.1.1.jar & Apache Commons Logging is a thin adapter allowing configurable bridging to other, well known logging systems.\\
\hline 
cxf-api-2.2.10.jar & Apache CXF is an open source services framework.  \\
\hline 
cxf-common-utilities-2.2.10.jar & This package contains different classes for Apache CXF.\\
\hline 
cxf-rt-bindings-soap-2.2.10.jar & This package contains necessary files to use Apache CXF as well with the Simple Object Access Protocol (SOAP).\\
\hline 
cxf-rt-core-2.2.10.jar & This library contains the Apache CXF Runtime Core. \\
\hline 
derby.jar & Apache Derby is a lightweight database written in Java which can also be used as an embedded database. This library contains the necessary drivers for the database as well as the database management system itself.\\
\hline 
jms-1.1.jar & Java Message Service is an API to send and receive messages within a client and to control so called Message Oriented Middleware (MOM).\\
\hline 
jndi-1.2.1.jar & The Java Naming and Directory Interface is an API which provides methods for multiple naming and directory services. It can be used for example to register disposed files in a network and to allow other part of a Java program to use them for RMI calls.\\
\hline 
junit-4.5.jar & This jar-file contains the necessary classes for the JUnit-tests, which can be used to test automatically Java classes.\\
\hline 
log4j-1.2.15.jar & Apache log4j is a framework for the logging of messages, errors and exceptions in Java applications.\\
\hline 
servlet-api.jar & The Java Servlet API supplies protocols to let applications respond for example to HTTP requests.\\
\hline 
sigar-1.6.3.jar & Hyperic SIGAR (System Information Gatherer) provides a Java API to system inventory and monitoring data (Memory, CPU etc.). In addition to the Jar file, it is required to add corresponding platform-specific native libraries to the classpath, which can be downloaded from~\cite{HypericSigarWebsite}. Kieker's \dir{lib/} folder already includes such libraries for Linux/Windows for the x86~architecture (\file{libsigar-x86-linux.so} and \file{sigar-x86-winnt.[dll|lib]}.\\
\hline 
spring.jar & The spring framework delivers different methods and classes to make the handling with Java/Java EE easier.\\
\hline 
spring-web.jar & This library contains the web application context, multipart resolver, Struts support, JSF support and web utilities for the spring framework.\\
\hline 
\end{longtable}
\label{tabular:libraries}
\end{center}

\subsection{Troubleshooting}
\end{document}
 
