\chapter{Kieker Tools}\label{chp:Kieker-Tools}
	\section{Trace Analysis}
		\subsection{Textual Trace and Equivalence Class Representations}
			\subsubsection{Execution Traces}
			\subsubsection{Message Traces}
			\subsubsection{Trace Equivalence Classes}
		\subsection{Sequence Diagrams}
			\subsubsection{Deployment-Level Sequence Diagrams}
			\subsubsection{Assembly-Level Sequence Diagram}
		\subsection{Call Trees}
			\subsubsection{Trace Call Trees}
			\subsubsection{Aggregated Call Trees}
		\subsection{Dependency Graphs}
			\subsubsection{Container Dependency Graphs}
			\subsubsection{Component Dependency Graphs}
			\subsubsection{Operation Dependency Graphs}
			\subsubsection{Response Times}
		\subsection{HTML Output of the System Model}	
			
	\section{Kieker WebGUI}
			
		The \KiekerWebGUI{} is a JavaEE-based web application to assemble, control, and observe Kieker analyses. Although currently still in the beta state, it already provides a user and project management, a graphical editor, and a control interface for analyses. A cockpit, which can be used, for example, for the realtime monitoring of applications, is currently under development.

		Like \Kieker{}, the \KiekerWebGUI{} project is licensed under the Apache License, Version 2.0. 
		
		\subsection{Download and Installation}
		
			The application can be downloaded as \file{.zip} and \file{.tar.gz} file on the \Kieker{} website under \url{http://kieker-monitoring.net/download/}. Once downloaded and extracted, the directory structure in Figure~\ref{fig:webgui-binary-layout} should be visible. In order to start the web application on Jetty, a lightweight web server, execute the suitable start script for your operation system in the \file{bin} directory. Depending on the system, the start procedure can take several minutes. Once started, the web application is available under \url{http://localhost:8080/Kieker.WebGUI/login}.
			
			\begin{figure}[h!]
				\begin{graybox}
					\dirtree{%
						.1 \DirInDirTree{\KiekerWebGUIDir/}.
							.2 \DirInDirTree{bin/}\DTcomment{Start scripts for the \KiekerWebGUI{}}.
								.3 Kieker.WebGUI.bat.
								.3 Kieker.WebGUI.sh.
							.2 \DirInDirTree{lib/}\DTcomment{Libraries required to start the \KiekerWebGUI{}}.
								.3 \ldots.
							.2 \DirInDirTree{target/}.
								.3 Kieker.WebGUI-1.7.war\DTcomment{The web application archive containing the \KiekerWebGUI{}}.
					}
				\end{graybox}
				
				\caption{Directory structure and contents of \KiekerWebGUI{}'s binary distribution}
				\label{fig:webgui-binary-layout}
			\end{figure}
			
			\noindent
			The web application provides per default three users (Table~\ref{tab:webgui-default-users}), which can be used to log in. Further users can be created when logged in as administrator.
			
			\begin{table}[h!]
				\center
				
				\begin{tabular}{|c|c|}
					\hline
					Username & Password\\
					\hline
					\hline
					guest    & kieker\\
					user     & kieker\\
					admin    & kieker\\
					\hline
				\end{tabular}
			
				\caption{Default users in the \KiekerWebGUI{}}
				\label{tab:webgui-default-users}
			\end{table}
			
		\subsection{Quickstart Example}
		
			\NOTIFYBOX{
				For the quickstart example it is assumed that you are already logged into the \KiekerWebGUI{}, either as an user or as an administrator. You should also enable javascript and cookies, as both is necessary for the functionality of the application. 
			}
		
		\subsection{Detailed Introduction}
			
	\section{Supporting Tools}
		\subsection{Replay Monitoring Logs}
		
			Replays filesystem monitoring logs created by \KiekerMonitoringPart{}. Example applications are:
			\begin{compactitem}
				\item 
				Merging multiple directories containing monitoring data into a single output directory. 
				\item 
				Importing a filesystem monitoring log to another monitoring log, e.g., a database. Therefore, an appropriate \KiekerMonitoringPart{} configuration	file must be passed to the script.
				\item 
				Replaying a recorded filesystem monitoring log in real-time in order to simulate incoming monitoring data from a running system, e.g., via JMS. 
			\end{compactitem}

			\

			\noindent Main-class: {\small \class{kieker.tools.logReplayer.FilesystemLogReplayerStarter}}

			\paragraph*{Usage}\

				\setTextListing
				\begin{lstlisting}[gobble = 10]
					usage: kieker.tools.logReplayer.FilesystemLogReplayerStarter
					 --c,--monitoring.configuration <\path\to\monitoring.properties>
							Configuration to use for the Kieker monitoring instance

					 --i,--inputdirs <dir1 ... dirN>
							Log directories to read data from

						--ignore-records-after-date <yyyyMMdd-HHmmss>
							Records logged after this date (UTC timezone) are ignored
							(disabled by default).

						--ignore-records-before-date <yyyyMMdd-HHmmss>
							Records logged before this date (UTC timezone) are ignored
							(disabled by default).

					 --k,--keep-logging-timestamps <true|false>
							Replay the original logging timestamps (defaults to true)?)

					 --n,--realtime-worker-threads <num>
							Number of worker threads used in realtime mode (defaults to 1).

					 --r,--realtime <true|false>
							Replay log data in realtime?. 
				\end{lstlisting}

			\paragraph*{Example}\

				\noindent The following command replays the monitoring testdata included in the binary release to another directory:

				\setTextListing
				\begin{lstlisting}[gobble = 10, caption=Execution under UNIX-like systems]
					$\lstshellprompt{}$ $\textbf{bin/logReplay.sh}$
					  $\textbf{-\,-inputdirs}$ $\distributedTestdataDirDistro$ 
					  $\textbf{-\,-keep-logging-timestamps}$ $true$ 
					  $\textbf{-\,-realtime}$ $false$
				\end{lstlisting}
				\begin{lstlisting}[gobble = 10, caption=Execution under Windows]
					$\lstshellprompt{}$ $\textbf{logReplay.bat}$
					  $\textbf{-\,-inputdirs}$ $\distributedTestdataDirDistroWin$ 
					  $\textbf{-\,-keep-logging-timestamps}$ $true$ 
					  $\textbf{-\,-realtime}$ $false$
				\end{lstlisting}
		
		\subsection{Convert Monitoring Timestamps}
		
			The script converts \KiekerMonitoringPart{} logging timestamps, representing the number of nanoseconds since 1~Jan 1970 00:00 UTC, to a human-readable textual representation in the UTC and local timezones.\\
			
			\noindent Main-class: {\small \class{kieker.tools.loggingTimestampConverter.LoggingTimestampConverterTool}}

			\paragraph*{Usage}\

				\setTextListing
				\begin{lstlisting}[gobble = 10]
					usage: kieker.tools.loggingTimestampConverter.LoggingTimestampConverterTool -t
						   <timestamp1 ... timestampN>
					 --t,--timestamps <timestamp1 ... timestampN>
							List of timestamps (UTC timezone) to convert
				\end{lstlisting}

			\paragraph*{Example}\

				\noindent
				The following listing shows the command to convert two logging timestamps as well as the resulting output.

				\setTextListing
				\begin{lstlisting}[gobble = 10, caption=Execution under UNIX-like systems]
					$\lstshellprompt{}$ $\textbf{bin/convertLoggingTimestamp.sh}$ $\textbf{-\,-timestamps}$ 1283156545581511026 1283156546127117246 
					1283156545581511026: Mo, 30 Aug 2010 08:22:25 +0000 (UTC) (Mo, 30 Aug 2010 10:22:25 +0200 (local time))
					1283156546127117246: Mo, 30 Aug 2010 08:22:26 +0000 (UTC) (Mo, 30 Aug 2010 10:22:26 +0200 (local time))
				\end{lstlisting}

				\begin{lstlisting}[gobble = 10, caption=Execution under Windows]
					$\lstshellprompt{}$ $\textbf{convertLoggingTimestamp.bat}$ $\textbf{-\,-timestamps}$ 1283156545581511026 1283156546127117246 
					1283156545581511026: Mo, 30 Aug 2010 08:22:25 +0000 (UTC) (Mo, 30 Aug 2010 10:22:25 +0200 (local time))
					1283156546127117246: Mo, 30 Aug 2010 08:22:26 +0000 (UTC) (Mo, 30 Aug 2010 10:22:26 +0200 (local time))
				\end{lstlisting}
		
		\subsection{KAX Viz}
		
			The KAX Viz tool visualizes a \KiekerAnalysisPart{} pipe-and-filter configuration file (\file{.kax} file).\\

			\noindent Main-class: {\small \class{kieker.tools.KaxViz}}

			\paragraph*{Usage}\

				\setTextListing
				\begin{lstlisting}[gobble = 10]
					usage: kieker.tools.KaxViz -i <filename> [-svg <filename>]
					 --i,--input <filename>
							the analysis project file (.kax) loaded

					 --svg <filename>
							name of svg saved on close
				\end{lstlisting}
		
		\subsection{KAX Runner}
		
			The KAX Runner tool executes a \KiekerAnalysisPart{} pipe-and-filter configuration file (\file{.kax} file). \\

			\noindent Main-class: {\small \class{kieker.tools.KaxRun}}

			\paragraph*{Usage}\

				\setTextListing
				\begin{lstlisting}[gobble = 10]
					usage: kieker.tools.KaxRun -i <filename>
					 --i,--input <filename>
							the analysis project file (.kax) loaded
				\end{lstlisting}	
			
	\section{TSLib \& OPAD}
	