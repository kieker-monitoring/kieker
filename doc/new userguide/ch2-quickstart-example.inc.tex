\chapter{Quickstart Guide}\label{chp:Quickstart-Guide}

	This chapter contains just a quickstart guide and does not show the full potential of Kieker. You can find more details in the rest of this user guide.

	\section{Monitoring}

		\begin{itemize}
			\item Download Kieker and copy \mainJarWeaver{} from the \dir{dist} directory into the same directory as your jar file.
			\item Execute the following command.
			\setBashListing
			\begin{lstlisting}[gobble=8,caption=,linewidth=0.94\textwidth]
				#\lstshellprompt{}# java -javaagent:#\mainJarWeaver# -jar MyJar.jar
			\end{lstlisting} 
		\end{itemize}
	
		\noindent
		Kieker writes the monitoring log files into the system’s default temporary directory (e.g. \dir{/tmp/}) in a directory named \dir{kieker-<date>-<timestamp>-UTC}. The precise path can be found in the console output.

	\section{Analysis}
	
		Under Linux you should use in the following the corresponding \file{.sh}-scripts instead of the \file{.bat}-scripts.
	

		\begin{itemize}
			\item Install Graphviz and make sure that the binaries are accessible via the system’s path.
			\item Execute the following command.
			\setBashListing
			\begin{lstlisting}[gobble=8,caption=,linewidth=0.94\textwidth]
				#\lstshellprompt{}# bin\trace-analysis.bat -i <temporary directory>/kieker-<date>-<timestamp>-UTC -o . --plot-Aggregated-Assembly-Call-Tree --plot-Assembly-Component-Dependency-Graph
			\end{lstlisting} 
			\item Execute the following command.
			\setBashListing
			\begin{lstlisting}[gobble=8,caption=,linewidth=0.94\textwidth]
				#\lstshellprompt{}# bin\dotPic-fileConverter.bat . png
			\end{lstlisting} 
		\end{itemize}	
		
		\noindent
		With these commands, Kieker reads the monitoring data logs and produces two graphs visualizing parts of the monitored application. 
		