This chapter gives a brief description on how to use the \class{AsyncJMSWriter} and \class{JMSReader} %
classes. This example is based on the Bookstore %
application with manual instrumentation presented in Chapter~\ref{chap:example}. %
The directory \dir{\JMSBookstoreApplicationDirDistro/} contains the %
sources, ant scripts etc. 

% \paragraph*{Preparation}

\begin{compactenum}
\item Copy the files \file{\mainJar} and \file{\commonsLoggingJar} from the %
binary distribution to the example's \dir{lib/} directory.
\item The file \file{\JMSBookstoreApplicationDirDistro/META-INF/kieker.monitoring.\-pro\-perties} %
is already configured to use the \class{AsyncJMSWriter}:
\end{compactenum}

\setPropertiesListing
\lstinputlisting[firstline=40,lastline=42,caption=Excerpt from \file{kieker.monitoring.properties} configuring the JMS writer]{\JMSBookstoreApplicationDir/META-INF/kieker.monitoring.properties}

\begin{compactenum}\setcounter{enumi}{2}
\item Download an OpenJMS install archive from \url{http://openjms.sourceforge.net} %
and decompress it to the root directory of the example. 
\item Copy the following files from the OpenJMS \dir{lib/} folder to the \dir{lib/} directory 
   of this example:
\begin{compactenum}
\item \file{openjms-<version>.jar}
\item \file{openjms-common-<version>.jar}
\item \file{openjms-net-<version>.jar}
\item \file{jms-<version>.jar}
\item \file{concurrent-<version>.jar}
\item \file{spice-jndikit-<version>.jar}
\end{compactenum}
\end{compactenum}

\enlargethispage{2cm}

% \paragraph*{Execution}%
 The execution of the example is performed by the following three steps:
\begin{enumerate}
\item Start the JMS server (you may have to set your \class{JAVA\_HOME} variable first):

\setBashListing
\begin{lstlisting}[caption=]
#\lstshellprompt{}# openjms-<version>/bin/startup.sh
\end{lstlisting}
\item Start the analysis part (in a new terminal):
\setBashListing
\begin{lstlisting}[caption=]
#\lstshellprompt{}# ant run-analysis
\end{lstlisting}
\item Start the instrumented Bookstore (in a new terminal):
\setBashListing
\begin{lstlisting}[caption=]
#\lstshellprompt{}# ant run-monitoring
\end{lstlisting}
\end{enumerate}